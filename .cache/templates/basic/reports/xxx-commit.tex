% Template fragment (basic, timeline-ready)
\section{提交考古:{curr.short}}

\subsection*{Commit 元信息}
\begin{itemize}
  \item 标题:{title}
  \item 作者:{curr.author}
  \item 日期:{curr.datetime}
\end{itemize}

\subsection*{变更摘要(阅读提示)}
% 建议:从 HEAD.diff 的开头几行(包含 diffstat)手动摘取 1–3 行,帮助读者把握范围。
\begin{verbatim}
HEAD:   <diffstat-from-HEAD.diff>
HEAD-1: <diffstat-from-HEAD-1.diff>
HEAD-2: <diffstat-from-HEAD-2.diff>
\end{verbatim}

\subsection*{差异解读(证据)}
% 结合 HEAD.diff / HEAD-1.diff / HEAD-2.diff,分点说明改了什么、为何而改、影响何在
\begin{itemize}
  \item 改了什么:<文件:行段>,<接口/结构/算法> 的关键变化。
  \item 为什么改:<性能/正确性/维护性> 的权衡与动机。
  \item 影响何在:对 <调用路径/构建/边界条件> 的影响与潜在风险。
  \item 如何验证:最小验证步骤(构建/测试/样例/基准)。
\end{itemize}

\subsection*{演进脉络}
% 从 head-2 → head-1 → head 的动机与取舍(若有)
\begin{itemize}
  \item head-2 → head-1:<要点/取舍>。
  \item head-1 → head:<要点/取舍>。
\end{itemize}

\subsection*{证据摘录(文件+行区间)}
% 精选 2–3 处证据,直指接口/数据结构/算法/边界
\begin{itemize}
  \item <path/to/file.rs:Lxx-Lyy>:<要点说明>。
  \item <path/to/other.rs:Laa-Lbb>:<要点说明>。
\end{itemize}

\subsection*{基础知识补充(计算几何)}
% 打开《计算几何教材.md》按关键词(orient2d/incircle/pseudo-angle/CDT 等)阅读,不超过 200 字
\noindent <阅读路径与结论:不超过 200 字,引用相关 HEAD*.diff 证据>

\subsection*{图示与说明(必选)}
% 必须提供两张图:架构图 + 算法流程图,并分别给出 3–5 句说明。
\begin{figure}[h]
  \centering
  \includegraphics[width=0.4\linewidth]{<NNN>-<{curr.short}>/architecture.pdf}
  \caption{架构变化(before/after、模块与依赖、数据路径、变化标注)}
\end{figure}

\noindent 架构图说明:
\begin{itemize}
  \item 变化范围:<受影响模块/边界>;
  \item 依赖与数据路径:<关键依赖/STEP→三角化→渲染链路>;
  \item 证据:引用 <HEAD*.diff> 中的 <文件:行段>;
  \item 风险:<兼容性/耦合度/部署影响>。
\end{itemize}

\begin{figure}[h]
  \centering
  \includegraphics[width=0.4\linewidth]{<NNN>-<{curr.short}>/algorithm_flow.pdf}
  \caption{算法流程(入口、关键分支/循环、边界处理、终止/复杂度)}
\end{figure}

\noindent 算法流程说明:
\begin{itemize}
  \item 入口与阶段:<主流程/子过程>;
  \item 分支与边界:<退化/数值鲁棒性处理、orient2d/incircle 调用点>;
  \item 复杂度与终止:<复杂度变化/终止条件>;
  \item 证据:对应 <HEAD*.diff> 的关键片段。
\end{itemize}
